%%%%%%%%%%%%%%%%%%%%%%%%%%%%%%%%%%%%%%%%%%%%%%%%%%%%%%%%%%%%%%%%%%%%%%%%
%                                                                      %
%     File: Thesis_Glossary.tex                                        %
%     Tex Master: Thesis.tex                                           %
%                                                                      %
%     Author: Andre C. Marta                                           %
%     Last modified : 30 Oct 2012                                      %
%                                                                      %
%%%%%%%%%%%%%%%%%%%%%%%%%%%%%%%%%%%%%%%%%%%%%%%%%%%%%%%%%%%%%%%%%%%%%%%%
%
% The definitions can be placed anywhere in the document body
% and their order is sorted by <symbol> automatically when
% calling makeindex in the makefile
%
% The \glossary command has the following syntax:
%
% \glossary{entry}
%
% The \nomenclature command has the following syntax:
%
% \nomenclature[<prefix>]{<symbol>}{<description>}
%
% where <prefix> is used for fine tuning the sort order,
% <symbol> is the symbol to be described, and <description> is
% the actual description.

% ----------------------------------------------------------------------

\glossary{name={\textbf{AUG}},description={The ASDEX Upgrade (Axially Symmetric Divertor EXperiment is the machine where the work of this thesis was carried out.}}

\glossary{name={\textbf{LFS}},description={Low Field Side, the outboard side of the tokamak vessel, where magnetic field is lower due to its $1/R$ dependance.}}

\glossary{name={\textbf{HFS}},description={High Field Side, the inboard side of the tokamak vessel.}}

\glossary{name={\textbf{RF}},description={Radio Frequency is a term normally employed for electromagnetic waves in the centimeter and milimiter wavelength.}}

\glossary{name={\textbf{ECRH}},description={Electron Cyclotron Resonant Heating is an RF heating method to couple energy to electrons at a selected cutoff location.}}

\glossary{name={\textbf{ICRH}},description={Ion Cyclotron Resonant Heating is an RF heating method to couple energy to ions.}}

\glossary{name={\textbf{NBI}},description={Neutral Beam Injection is a heating method that launches energetic neutral atoms at the confined plasma.}}

\glossary{name={\textbf{DoD}},description={The Degree of Detachment is an experimental value that characterizes how far has the detachment state of a divertor has progressed..}}

\glossary{name={\textbf{AS}},description={Attached State, when a divertor is fully attached.}}

\glossary{name={\textbf{OS}},description={Onset State, when the detachment process commences and the measured ion flux at the target deviates from a scaling based on edge density.}}

\glossary{name={\textbf{FS}},description={Fluctuating State is a detachment state when fluctuations can be seen close to the C-point..}}

\glossary{name={\textbf{CDS}},description={Complete Detachment State is when the divertor can be characterized as fullhy detached based on ion flux measurements.}}

\glossary{name={\textbf{SOL}},description={The Scrape-Off Layer is the plasma region outside the separatrix, characterized by open field lines.}}

\glossary{name={\textbf{HFSHD}},description={The High Field Side High Density is a region at the HFS inner divertor that can extend up to the midplane with densities comparable to the ones of the confined plasma.}}

\glossary{name={\textbf{ELM}},description={Edge Localized Modes are transient relaxations of the confined plasma pressure gradient responsible for large expulsions of particles and energy.}}



